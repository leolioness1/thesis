%!TEX root = ../template.tex
%%%%%%%%%%%%%%%%%%%%%%%%%%%%%%%%%%%%%%%%%%%%%%%%%%%%%%%%%%%%%%%%%%%%
%% abstract-pt.tex
%% NOVA thesis document file
%%
%% Abstract in Portuguese
%%%%%%%%%%%%%%%%%%%%%%%%%%%%%%%%%%%%%%%%%%%%%%%%%%%%%%%%%%%%%%%%%%%%

\typeout{NT FILE abstrac-pt.tex}
\paragraph{}
% O aquecimento global tem sido tópico de discussão nas últimas décadas. Apesar deste debate, o impacto do aquecimento global no degelo do pergelissolo e vice-versa não está amplamente estudado nem documentado. Uma das causas que pode ter levado a esta escassez de estudos é o facto do pergelissolo se encontrar no Ártico ou em regiões igualmente remotas e inacessiveis, o que faz com que a recolha de dados seja difícil e com custos elevados.

% Uma das soluções parciais para este problema, usada há várias décadas, é a recolha de imagens de satélite para estudar as mudanças nas regiões de pergelissolo. Apesar dos inúmeros benefícios, esta técnica requer uma análise detalhada das imagens adquiridas, o que por consequinte se traduz num processo exaustivo e demorado quando é feito manualmente por cientistas. 

% Ao longo da últimas década, o crescimento de “Deep Learning” propoe resolver estas limitacoes. O uso desta ferramenta para a análise de imagens de satélite tem crescido em popularidade, em particular devido ao aumento da quantidade e disponibilidade de dados. Este aumento de dados tem sido sustentado em grande parte pela desponibilização, na modalidade de "open-source" de dados de sensores multiespectrais de alta resolução espacial, como aqueles usados neste projeto, provenientes da missão “Sentinel-2”.


% No entanto, apesar de um crescimento do uso de “Deep Learning” na analise de imagens de satélite a sua aplicação concreta  específicamente na análise do degelo do pergelissolo, abordada nest projecto, não tem sido amplamente estudado. Para abordar esta lacuna, o modelo de “semantic segmentation” proposto neste projeto, classifica cada pixel nas imagens de satélite para identificar "Retrogressive Thaw Slumps (RTSs)”, usando a arquitetura “U-Net”.

% % Com o uso desta técnica, este projeto visa ajudar na identificação destas formas de relevo de uma maneira automatizada e em larga escala. 
% Neste projecto, a identificação de RTSs usando imagens de satélite é bem sucedida, conseguindo um "Dice Score" médio de $95\%$, nas 39 images de teste analisadas. Este resultado levou a conclusao que é possivél processar imagens de satélite e atingir resultados satisfatorios usando imagens com 10 metros de resolução espacial e apenas 4 bandas espectrais. Como estas formas de relevo são uma boa indicação do degelo do pergelissolo, a esperança é que este projeto possa ajudar na mitigação do impacto deste poderoso fenómeno geofísico.

O aquecimento global tem sido tópico de discussão nas últimas décadas. Apesar deste debate, o impacto do aquecimento global no degelo do pergelissolo e vice-versa não está amplamente estudado nem documentado. Uma das causas que pode ter levado a esta escassez de estudos é o facto do pergelissolo se encontrar no Ártico ou em regiões igualmente remotas e inacessíveis, o que faz com que a recolha de dados seja difícil e com custos elevados.

Uma das soluções parciais para este problema, usada há várias décadas, é a recolha de imagens de satélite para estudar as mudanças nas regiões de pergelissolo. Apesar dos inúmeros benefícios, esta técnica requer uma análise detalhada das imagens adquiridas, o que, por conseguinte, se traduz num processo exaustivo e demorado quando é feito manualmente por cientistas. 

Ao longo das últimas décadas, o crescimento de “Deep Learning” propõe resolver estas limitações. O uso desta ferramenta para a análise de imagens de satélite tem crescido em popularidade, em particular devido ao aumento da quantidade e disponibilidade de dados. Este aumento de dados tem sido sustentado em grande parte pela disponibilização, na modalidade de “open-source” de dados de sensores multiespectrais de alta resolução espacial, como aqueles usados neste projeto, provenientes da missão “Sentinel-2”.

No entanto, apesar de um crescimento do uso de “Deep Learning” na análise de imagens de satélite a sua aplicação concreta especificamente na análise do degelo do pergelissolo, abordada neste projeto, não tem sido amplamente estudado. Para abordar esta lacuna, o modelo de “semantic segmentation” proposto neste projeto, classifica cada pixel nas imagens de satélite para identificar "Retrogressive Thaw Slumps (RTSs)”, usando a arquitetura “U-Net”.

% Com o uso desta técnica, este projeto visa ajudar na identificação destas formas de relevo de uma maneira automatizada e em larga escala. 

Neste projeto, a identificação de RTSs usando imagens de satélite é bem sucedida, conseguindo um “Dice Score” médio de $95\%$, nas 39 imagens de teste analisadas. Este resultado levou a conclusão que é possível processar imagens de satélite e atingir resultados satisfatórios usando imagens com 10 metros de resolução espacial e apenas 4 bandas espectrais. Como estas formas de relevo são uma boa indicação do degelo do Pergelissolo, a esperança é que este projeto possa ajudar na mitigação do impacto deste poderoso fenómeno geofísico.

% Palavras-chave do resumo em Português
\begin{keywords}
Pergelissolo; Retrogressive Thaw Slump (RTS); Imagens de Satélite; Multispectral Instrument; Machine Learning (ML); Deep Learning (DL); Artificial Intelligence (AI); U-Net; Redes Neuronais Convolucionais; Visão Computacional.
\end{keywords}

% to add an extra black line
% \begin{enumerate}
%   \item Qual é o problema?
%   \item Porque é que é um problema interessante/desafiante?
%   \item Qual é a proposta de abordagem/solução?
%   \item Quais são as consequências/resultados da solução proposta?
% \end{enumerate}
% E agora vamos fazer um teste com uma quebra de linha no hífen a ver se a \LaTeX\ duplica o hífen na linha seguinte se usarmos \verb+"-+… em vez de \verb+-+.
%
% zzzz zzz zzzz zzz zzzz zzz zzzz zzz zzzz zzz zzzz zzz zzzz zzz zzzz zzz zzzz comentar"-lhe zzz zzzz zzz zzzz
%
% Sim!  Funciona! :)