%!TEX root = ../template.tex
%%%%%%%%%%%%%%%%%%%%%%%%%%%%%%%%%%%%%%%%%%%%%%%%%%%%%%%%%%%%%%%%%%%%
%% abstract-pt.tex
%% NOVA thesis document file
%%
%% Abstract in Portuguese
%%%%%%%%%%%%%%%%%%%%%%%%%%%%%%%%%%%%%%%%%%%%%%%%%%%%%%%%%%%%%%%%%%%%

\typeout{NT FILE abstrac-pt.tex}
\paragraph{}
Nas últimas décadas o aquecimento global tem sido tópico de discussão. Contudo o seu impacto no derretimento de “permafrost” (solo permanentemente congelado) e vice-versa não tem sido bem registado no passado. Uma das razões que pode ter levado a isto é o facto do “permafrost” se encontrar no Ártico ou em regiões igualmente remotas de difícil acesso, o que faz com que a recolha de dados seja difícil e com demasiados custos.


Uma das soluções para este problema é a recolha de imagens de satélite para analisar as mudanças nas regiões de “permafrost”, onde já é usada há muitas décadas. Apesar dos seus inúmeros benefícios, esta técnica requer inspeção e análise detalhada das imagens, e quando é feita manualmente pelos cientistas pode tornar-se monótona e demorada.

Nos últimos anos, o uso de “Deep Learning” para a análise de imagens de satélite tem aumentado, este popularismo deve-se ao aumento da quantidade e disponibilidade das mesmas. Os dados obtidos por esta tecnologia têm sido cada vez mais relevantes devido ao uso de sensores multiespectrais de alta resolução espacial, como aqueles usados neste projeto, provenientes da missão “Sentinel-2”.


No entanto, o uso de “Deep Learning” específicamente para a identificação de “permafrost” não tem sido amplamente estudado. Para combater este problema, o modelo de “semantic segmentation” proposto neste projeto, classifica cada pixel nas imagens de satélite para identificar "Retrogressive Thaw Slumps (RTSs)”, usando a arquitetura “U-Net”.

Com o uso desta técnica, este projeto visa ajudar na identificação destas formas de relevo de uma maneira mais automática e numa escala maior. Como estas formas de relevo são uma boa indicação do derretimento de “permafrost”, a esperança é que este projeto possa ajudar na mitigação do impacto deste poderoso fenómeno geofísico.

% Palavras-chave do resumo em Português
\begin{keywords}
  Permafrost; Retrogressive Thaw Slump (RTS); Imagens de satelite;  Multispectral Instrument; Machine Learning (ML); Deep Learning (DL); Artificial Intelligence (AI);; U-Net; Redes Neuronais Con-
  volucionais; Visão computacional \ldots
\end{keywords}

% to add an extra black line
% \begin{enumerate}
%   \item Qual é o problema?
%   \item Porque é que é um problema interessante/desafiante?
%   \item Qual é a proposta de abordagem/solução?
%   \item Quais são as consequências/resultados da solução proposta?
% \end{enumerate}
% E agora vamos fazer um teste com uma quebra de linha no hífen a ver se a \LaTeX\ duplica o hífen na linha seguinte se usarmos \verb+"-+… em vez de \verb+-+.
%
% zzzz zzz zzzz zzz zzzz zzz zzzz zzz zzzz zzz zzzz zzz zzzz zzz zzzz zzz zzzz comentar"-lhe zzz zzzz zzz zzzz
%
% Sim!  Funciona! :)