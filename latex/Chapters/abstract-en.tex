%!TEX root = ../template.tex
%%%%%%%%%%%%%%%%%%%%%%%%%%%%%%%%%%%%%%%%%%%%%%%%%%%%%%%%%%%%%%%%%%%%
%% abstrac-en.tex
%% NOVA thesis document file
%%
%% Abstract in English
%%%%%%%%%%%%%%%%%%%%%%%%%%%%%%%%%%%%%%%%%%%%%%%%%%%%%%%%%%%%%%%%%%%%
\paragraph{}
Global warming has been a topic of discussion for many decades, however its impact on the thaw of permafrost (frozen ground) and vice-versa has not been very well captured or documented in the past. This may be due to most of permafrost being in the Arctic and similarly vast remote areas, which makes data collection difficult and costly.

A partial solution to this problem is the use of Remote Sensing imagery, which isn't novel in documenting the changes in permafrost regions, where it has been widely used for decades. However, this still required manual inspection of images which could be a slow tedious task for researchers in this field.

Over the last decade, the use of Deep Learning algorithms for Remote Sensing imagery has risen in popularity, mainly due to the increased availability and scale of Remote Sensing data. This has been fuelled in the last few years by open-source multi-spectral high spatial resolution data, such as the Sentinel-2 data used in this project.
The use of Deep Learning for the particular use case of identifying the thaw of permafrost addressed in this project has nonetheless not been widely studied.

The model proposed in this project aims to aid in the identification of landforms that can be a proxy for the thaw of permafrost, in the hope it can make progress towards the mitigation of the impact of such a powerful geophysical phenomenon.

% Palavras-chave do resumo em Inglês
\begin{keywords}
Permafrost; \gls{RTS}; Remote Sensing;  Multispectral Instrument; \gls{ML}; \gls{DL}; Artificial Intelligence; \gls{CNN}; \gls{CV} \ldots
\end{keywords} 
