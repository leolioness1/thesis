%!TEX root = ../template.tex
%%%%%%%%%%%%%%%%%%%%%%%%%%%%%%%%%%%%%%%%%%%%%%%%%%%%%%%%%%%%%%%%%%%%
%% abstrac-en.tex
%% NOVA thesis document file
%%
%% Abstract in English
%%%%%%%%%%%%%%%%%%%%%%%%%%%%%%%%%%%%%%%%%%%%%%%%%%%%%%%%%%%%%%%%%%%%
\paragraph{}
Global warming has been a topic of discussion for many decades, however its impact on the thaw of permafrost and vice-versa has not been very well captured or documented in the past. This may be due to most permafrost being in the Arctic and similarly vast remote areas, which makes data collection difficult and costly.

A partial solution to this problem is the use of Remote Sensing imagery, which has been widely used for decades in documenting the changes in permafrost regions. Despite its many benefits, this methodology still required a manual assessment of images, which could be a slow and laborious task for researchers.
Over the last decade, the growth of Deep Learning has helped address these limitations. The use of Deep Learning on Remote Sensing imagery has risen in popularity, mainly due to the increased availability and scale of Remote Sensing data. This has been fuelled in the last few years by open-source multi-spectral high spatial resolution data, such as the Sentinel-2 data used in this project.

Notwithstanding the growth of Deep Learning for Remote Sensing Imagery, its use for the particular use case of identifying the thaw of permafrost, addressed in this project, has not been widely studied. To address this gap, the semantic segmentation model proposed in this project performs pixel-wise classification on the satellite images for the identification of Retrogressive Thaw Slumps (RTSs), using a U-Net architecture.

% By using this technique, this project aims to aid the identification of these landforms at a larger scale and in a more automated way
In this project, the successful identification of RTSs using Satellite Images is achieved with an average of $95\%$ Dice score for the 39 test images evaluated, concluding that it is possible to pre-process said images and achieve satisfactory results using 10-meter spatial resolution and as little as $4$ spectral bands. Since these landforms can be a proxy for the thaw of permafrost, the aim is that this project can help make progress towards the mitigation of the impact of such a powerful geophysical phenomenon.


\begin{keywords}
Permafrost; Retrogressive Thaw Slump (RTS); Remote Sensing;  Multispectral (MSI); Machine Learning (ML); Deep Learning (DL); Artificial Intelligence (AI); U-NET; Convolutional Neural Network (CNN); Computer Vision (CV) \ldots
\end{keywords} 
