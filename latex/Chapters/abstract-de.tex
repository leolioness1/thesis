%!TEX root = ../template.tex
%%%%%%%%%%%%%%%%%%%%%%%%%%%%%%%%%%%%%%%%%%%%%%%%%%%%%%%%%%%%%%%%%%%%
%% abstrac-de.tex
%% NOVA thesis document file
%%
%% Abstract in English
%%%%%%%%%%%%%%%%%%%%%%%%%%%%%%%%%%%%%%%%%%%%%%%%%%%%%%%%%%%%%%%%%%%%

\typeout{NT FILE abstrac-de.tex}

\textbf{This is a “Google Translate” translation from the English version!  Fixes and corrections are welcome!}

Die Dissertation muss zwei Versionen des Abstracts enthalten, eine in derselben Sprache wie der Haupttext und eine in einer anderen Sprache. Das Paket geht davon aus, dass die beiden betrachteten Sprachen immer die Hauptsprache und Englisch sind. Und wenn die Hauptsprache Englisch ist, werden Englisch und Portugiesisch vorausgesetzt. Sie können dieses Verhalten durch Hinzufügen ändern
\begin{verbatim}
    \abstractorder(<MAIN_LANG>):={<LANG_1>,...,<LANG_N>}
\end{verbatim}
\noindent e.g.,
\begin{verbatim}
    \abstractorder(de):={de,en,it}
\end{verbatim}

Das Paket sortiert die Abstracts in der entsprechenden Reihenfolge. Dies bedeutet, dass das erste Abstract in derselben Sprache wie der Haupttext vorliegt, gefolgt vom Abstract in der anderen Sprache und anschließend vom Haupttext. Wenn die Dissertation beispielsweise in Portugiesisch verfasst ist, wird zuerst die Zusammenfassung in Portugiesisch und dann in Englisch angezeigt, gefolgt vom Haupttext in Portugiesisch. Wenn die Dissertation in englischer Sprache verfasst ist, wird zuerst die Zusammenfassung in englischer und dann in portugiesischer Sprache angezeigt, gefolgt vom Haupttext in englischer Sprache.

Das Abstract sollte eine Seite nicht überschreiten und die folgenden Fragen beantworten:

\begin{itemize}
  \item Was ist das Problem?
  \item Warum ist es interessant?
  \item Was ist die Lösung?
  \item Was folgt aus der Lösung?
\end{itemize}

% Palavras-chave do resumo em Inglês
\begin{keywords}
Schlüsselwort 1, Schlüsselwort 2, Schlüsselwort 3, …
\end{keywords} 
